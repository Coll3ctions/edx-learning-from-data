\documentclass{article}

\usepackage{amsmath}
\usepackage{numprint}

\author{Daniel Fernandes Martins (danielfmt)}
\title{Question \#1 Solution}

\begin{document}

\maketitle

\textbf{Disclaimer.} This is the reasoning I used to solve the problem; it
may be wrong though. This is intended just as food for thought.

\section{Deterministic Noise Behavior}

Assuming $\mathcal{H}' \subset \mathcal{H}$ and that $f$ is fixed, in general,
deterministic noise tends to \textbf{increase} because $\mathcal{H}'$ is not
powerful enough to approximate $f$, or at least not as powerful as its
superset $\mathcal{H}$.

Let's remember that \textbf{stochastic noise} is fixed for a given dataset
$\mathcal{D}$ regarless of your choice of $\mathcal{H}$, which means it doesn't
vary from one $\mathcal{H}$ to another. The \textbf{deterministic noise}, on the
other hand, do vary due to the fact that the portion that cannot be captured by
a particular $\mathcal{H}'$ is considered noise.

\end{document}
