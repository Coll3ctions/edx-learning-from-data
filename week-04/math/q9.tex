\documentclass{article}

\usepackage{amsmath}
\usepackage{numprint}

\author{Daniel Fernandes Martins (danielfmt)}
\title{Question \#9 Solution}

\begin{document}

\maketitle

\textbf{Disclaimer.} This is the reasoning I used to solve question \#10; it
may be wrong though. This is intended just as food for thought.

\section{Finding The Lower Bound}

Let's start with two hypothesis sets with break point $k=2$ and $N=2$. This
means that these hypothesis sets cannot shatter any combination of 2 points,
which implies that $d_{vc}=1$ for both of them.

These are the dichotomies realized by $\mathcal{H}_1$ on these 2 points:

\begin{equation*}
\begin{split}
\{a=-1, b=+1\} \\
\{a=+1, b=-1\}
\end{split}
\end{equation*}

For $\mathcal{H}_2$:

\begin{equation*}
\begin{split}
\{a=-1, b=-1\} \\
\{a=+1, b=+1\}
\end{split}
\end{equation*}

Notice that the dichotomies realized by $\mathcal{H}_1$ are different than the
ones realized by $\mathcal{H}_2$, which means the intersection
$\mathcal{H}_1\bigcap\mathcal{H}_2$ is an empty set, so $d_{vc}=0$.

\section{Finding The Upper Bound}

Let's say you have two hypothesis sets $\mathcal{H}_1$ and $\mathcal{H}_2$. The
biggest set you can get from any intersection of these two sets is by having
two sets such that $\mathcal{H}_2\subseteq\mathcal{H}_1$. Therefore, the upper
bound can be at most as big as the smallest $\mathcal{H}_k$.

\section{Solution}

\begin{equation*}
0 \leq \\
  d_{vc}(\bigcap_{k=1}^K\mathcal{H}_k) \leq \\
  \min\{d_{vc}(\mathcal{H}_k)\}_{k=1}^K
\end{equation*}

\end{document}
