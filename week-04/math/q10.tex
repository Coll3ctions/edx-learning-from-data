\documentclass{article}

\usepackage{amsmath}
\usepackage{numprint}

\author{Daniel Fernandes Martins (danielfmt)}
\title{Question \#10 Solution}

\begin{document}

\maketitle

\textbf{Disclaimer.} This is the reasoning I used to solve the problem; it
may be wrong though. This is intended just as food for thought.

\section{Finding The Lower Bound}

Let's say we have the hypothesis sets $\mathcal{H}_1$ and $\mathcal{H}_2$,
with VC dimensions $d_{vc}(\mathcal{H}_1)=2$ and $d_{vc}(\mathcal{H}_2)=3$.
This means that $\mathcal{H}_1$ shatters at most $2$ points, and
$\mathcal{H}_2$ shatters at most $3$ points.

If we join these two hypothesis sets in a single $\mathcal{H}$, we are
guaranteed to shatter at least $3$ points. So, the lower bound is the
maximum number of points that can be shattered by any $\mathcal{H}_k$.

\section{Finding The Upper Bound}

We are given two upper bound candidates, $\sum_{k=1}^Kd_{vc}(\mathcal{H}_k)$
or $K-1+\sum_{k=1}^Kd_{vc}(\mathcal{H}_k)$.

\subsection {Intuition on $\sum_{k=1}^Kd_{vc}(\mathcal{H}_k)$}

Let's start with two hypothesis sets with break point $k=2$ and $N=2$. This
means that these hypothesis sets cannot shatter any combination of 2 points,
which implies that $d_{vc}=1$ for both of them.

These are the dichotomies realized by $\mathcal{H}_1$ on these 2 points:

\begin{equation*}
\begin{split}
\{a=-1, b=+1\} \\
\{a=+1, b=-1\}
\end{split}
\end{equation*}

For $\mathcal{H}_2$:

\begin{equation*}
\begin{split}
\{a=-1, b=-1\} \\
\{a=+1, b=+1\}
\end{split}
\end{equation*}

The union hypothesis set $\mathcal{H}_1\bigcup\mathcal{H}_2$ have $d_{vc}=2$,
so the option where the upper bound is the sum of $d_{vc}$ over all
$\mathcal{H}_k$ seems a reasonable choice, at least for these two hypothesis
sets.

\subsection{Intuition on $K-1+\sum_{k=1}^Kd_{vc}(\mathcal{H}_k)$}

Now let's take $N=3$ points and two new hypothesis sets,
$\mathcal{H}_3$ and $\mathcal{H}_4$. What if we could divide all eight
dichotomies carefully between these two hypothesis sets so that both
hypothesis sets have $d_{vc}=1$?

After moving the dichotomies around, this is $\mathcal{H}_3$:

\begin{equation*}
\begin{split}
\{a=-1, b=-1, c=-1\} \\
\{a=-1, b=-1, c=+1\} \\
\{a=-1, b=+1, c=-1\} \\
\{a=+1, b=-1, c=-1\}
\end{split}
\end{equation*}

This is $\mathcal{H}_4$:

\begin{equation*}
\begin{split}
\{a=-1, b=+1, c=+1\} \\
\{a=+1, b=-1, c=+1\} \\
\{a=+1, b=+1, c=-1\} \\
\{a=+1, b=+1, c=+1\}
\end{split}
\end{equation*}

Noticed something? Well, both hypothesis sets $\mathcal{H}_3$ and
$\mathcal{H}_4$ have $d_{vc}=1$, but the union hypothesis set
$\mathcal{H}_3\bigcup\mathcal{H}_4$ have $d_{vc}=3$. Also, since in this
example $K=2$, the bound seems to hold.

\subsubsection{Relation With $B(N,k)$}

The analysis definitely has to do with $B(N, k)$, although I don't know how
to prove it yet.

\section{Solution}

$$
\max\{d_{vc}(\mathcal{H}_k)\}_{k=1}^K \leq d_{vc}(\bigcup _{k=1}^K)\mathcal{H}_k \leq K-1+\sum_{k=1}^Kd_{vc}(\mathcal{H}_k)
$$

\end{document}
